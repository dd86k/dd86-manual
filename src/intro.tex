\chapter{Introduction}

The Microsoft Disk Operating System, mostly known as MS-DOS, derived from 86-DOS,
was the most used DOS-based operating system from its time, making it a key product for
Microsoft to grow as a large software company at the time.
\\\\
Introduced in Windows NT 3.51, and on future 32-bit WindowsNT versions of the Windows
operating system, the WindowsNT Virtual DOS Machine emulation layer, known as
NTVDM, is a software emulation layer to execute 16-bit DOS console applications within
the console window.
\\\\
However, since the release of 64-bit versions of the Windows operating system, the
NTVDM emulation layer had been stripped out of the system, due to x86-64 based
processors not supporting the legacy 16-bit virtual mode (vm8086) in 64-bit mode (LONG
mode).